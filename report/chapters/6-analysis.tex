\chapter{Analysis} \label{chap:analysis}

The purpose of these experiments was to investigate which traffic layout benefits traffic participants most. From the data shown above, it is clear that throughput in the large crosswalk scenario is much more stable than in the small crosswalk scenario, both across runs and agent types. This can be seen in its overall lower and more monotonic variance. While this in itself does not yield conclusions on performance, it indicates that those travelling across this scenario can rely on a more predictable amount of time that they will need to get across.

The average throughput indicates another metric in which the large crosswalk surpasses the small crosswalk: For both pedestrians and cars, the time an agent requires to reach their target is reduced in the large crosswalk setting. This speaks in favour of the shared-space (large crosswalk) model, as its efficiency seems higher than its counterpart (small crosswalk).

In general, although the large crosswalk scenario does seem to perform better and more predictably, it must be noted that these differences only play a role in busy situations. As traffic load reduces, the differences become almost negligible. This indicates that these findings are likely to only be relevant for traffic sites that will be frequented by many participants simultaneously, and less so for less busy urban locations.

However, as with any computer model, there are several threats to the validity of these conclusions. This simulator has not been validated against real-world scenarios, due to us not having access to the kind of real-world measurement traces needed to compare our findings with realistic measurements of throughput. Only relying on visual verification and programmatic assertions does not guarantee that the simulation results are realistic. The simplicity of our model is another concern brought up by only using simulations. A model typically needs to simplify some aspects of a problem in order to be computable, ours is no different. We did not take into account social interactions such as eye contact or differing characteristics of participants. The simulation also assumes uniform speed across all cars, and instantaneous acceleration and breaking, which is also why collisions or near-miss scenarios are not simulated in this model. Although we believe all these parameters to be less crucial to the simulation than what is already in the model, it is difficult to predict what their influence would have been on the results that were obtained.

