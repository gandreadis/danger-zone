\chapter{Experimental Setup} \label{chap:setup}

Our core set of experiments consists of two maps, both having two sidewalks and a two-lane road in the center. Pedestrian spawn points are on both ends of both sidewalks. To test our hypothesis, two modified versions of this map were made. The first has a small crosswalk area in the middle of the road, representing a traditional traffic situation in which pedestrians need to cross the road at a bottleneck to continue their journey. Map 2 has a significantly wider crosswalk area, representing a shared-space model for traffic. Both maps are defined in .dzone files, and can be seen in a folder on our GitHub repository\footnote{\url{https://github.com/gandreadis/danger-zone/tree/master/maps}}.

Command-line arguments are included in order to automate the running of multiple simulations, as well as set the number of ticks each will run for, and set the spawn delays for both cars and pedestrians.  Each of these were set and run for both maps, and a command line script running the entire set of parameter combinations tested can be found in the repository\footnote{\url{https://github.com/gandreadis/danger-zone/blob/master/experiment_scripts/straight_crosswalk_experiment.cmd}}.

\section{Number of Ticks}
The simulation is tick-based, meaning that time is discretized into distinct intervals, and progress is evaluated at each of these. Inherent to any time-based simulation is the need to restrict its duration, under a certain exit criterion. In this series of experiments, the exit criterion was chosen to be 1000 ticks. Our initial explorations of parameter combinations indicated that this would strike a balance between conserving computation time and allowing more data and emergent behaviours to be produced.

\section{Pedestrian and Car Spawn Delays}
Agents are spawned periodically as a simulation runs, and as such these periods (i.e. the number of ticks between two spawns of a certain type) can be adjusted by command-line arguments in order to simulate different types of traffic load. Two series of runs were performed on each map: One with a fixed car spawn delay and increasing pedestrian spawn delay and one that fixed the pedestrian delay and increased the spawn delay (in order to only vary one parameter at a time). The exact values and combinations can be found in the experiment script.

\section{Number of Repeated Iterations}
Due to the non-deterministic nature of the randomised spawns, different simulation runs with the same parameters could lead to different results. To ascertain that the results did not rely on unrepresentative data, each combination of the above 3 parameters was repeated 10 times.
