\chapter{Related Work} \label{chap:related}

Although research into traffic flow and modelling already exists, not much of it has been translated to publicly-available models or publications. What we did find available did not include any exact matches for our problem statement. For instance, the SUMO model, whilst expansive, lacks detail in its shared space implementation, and is overall too much data for our purposes\cite{krajzewicz2014}, and the Vissim model relies on unusual implementations (for example by defining vehicles as groups of pedestrians) that do not allow for much analysis\cite{kupferschmid2016}.

Early versions of our model used an agent system instead of a cellular automata approach. The flocking behaviours it allowed us to use were useful for representing traffic flow and keeping the agents together, and some of the base code from a The Nature of Code's article\cite{shiffman2012} was used to build our model. However, the autonomous agent system alone was not enough. Looking at existing papers, we sought inspiration for a cellular automata approach. We used elements of a model for pedestrian traffic\cite{min2017} that used a floor field cellular automaton model to determine responses to accidents, as well as a mixed traffic simulation that watched for pedestrian and vehicle behaviour at a crosswalk using a series of more complex sub-models and a much smaller map\cite{zhang2007} as references for our updated model.

For the `shared spaces' aspect of the project, existing real-life areas were considered. Of particular interest was Amsterdam's implementation of a shared bicycle/pedestrian space behind Centraal Station\cite{kruyswijk2016}. This presented a real-life equivalent of what the model was trying to simulate, providing inspiration on what is possible in the shared-space domain.
