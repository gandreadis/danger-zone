\chapter{Related Work} \label{chap:related}

There already exists much research into traffic flow and modelling, but not much of it has been translated to publicly-available models or papers, and there was no exact match for our research. For instance, the SUMO model, whilst expansive, lacked detail in its shared space implementation, and was overall too much data for our purposes \cite{krajzewicz2014}, and the Vissim model relied on strange implementations that didn’t allow for much analysis, such as by defining pedestrians as small vehicles \cite{kupferschmid2016}.
Early versions of our model used an agent system instead of a cellular automata approach. The flocking behaviours it allowed us to use were useful for representing traffic flow and keeping the agents together, and some of the base code from The Nature of Code's article on this \cite{shiffman2012} was used to build our model. However, the autonomous agent system was not enough on its own. Looking at existing papers, inspiration for a cellular automata approach was sought. Elements of a model for pedestrian traffic \cite{min2017} and used a floor field cellular automaton model to determine responses to accidents, as well as a mixed traffic simulation that watched for pedestrian and vehicle behaviour at a crosswalk using a series of more complex sub-models and a much smaller map \cite{zhang2007} were used as references for our updated model.
For the ‘shared spaces’ aspect of the project, existing real life areas were looked at. Of particular interest was Amsterdam's implementation of a shared bicycle/pedestrian space behind Centraal Station \cite{kruyswijk2016}. This presented a real-life equivalent of what the model was trying to simulate, providing inspiration on what is possible in the shared-space domain.
