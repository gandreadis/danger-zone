\chapter{Discussion} \label{chap:discussion}

If nothing else, this project made us realise the complexity of modeling real-life traffic scenarios. We set out to compare different traffic layouts from the viewpoint of efficiency (throughput), and to this aim two different approaches were attempted. The first approach, consisting of free-moving agents, falls short of being able to accurately model complex map layout navigations. The second approach, which discretized the map field and used CA-like rules for decision making, comes closer to our goal of comparative traffic layout evaluation. It is able to measure throughput in simple traffic scenarios, and can be used to compare map layouts on their efficiency. However, it fails to model collisions, preventing any measurement of the safety of layouts.

Our initial goal was ambitious, as we discovered during design and development. A system without discrete steps models the world more accurately, but is significantly more difficult to design. We faced this when trying to give the free agents in our first design some type of pathfinding ability. On the other hand, the discrete steps system we selected was easier to design and implement, but it also resembled the real world less. On top of this fundamental dilemma came various difficulties, from diverging levels of Python programming knowledge within the team to a general lack of time due to various circumstances and absences.

In the end, we were able to tentatively show superiority of a shared-space model over a traditional model. This is, however, just the beginning. There is still plenty of area to explore in this field, as evidenced by the large body of existing work in this domain. Most importantly perhaps in our case would be the addition of bicycle agents: Their agility and position in traffic makes them a unique aspect, and leaves many questions open about how they influence traffic patterns. After having added cyclists, more complex scenarios could be tested. Although the tested scenarios exist in real life, they are limited in scope and complexity, and have a great deal of possible expansion. More research could also go into the social interactions that take place in a traffic situation. Possible interactions that the model could benefit from would be visual contact (deciding who takes precedence) or use of sound signals (bells or horns on bikes or cars).

