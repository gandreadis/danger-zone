\begin{abstract}
    The rise of motorised travel has revolutionised how roads are used, and this can make travel challenging for pedestrians. One particular challenge has been the use of shared spaces - areas where pedestrians and other vehicles may need to utilise the road together, usually seen with pedestrian crossings. This type of urban traffic layout has seen a recent upsurge in popularity, with multiple municipalities opting to implement it in highly frequented traffic locations. These layouts are often claimed to increase efficiency and decrease risk of accidents, but  existing traffic models have failed to analyse their full consequences. This paper proposes a new model, called ‘Danger Zone’, which simulates some basic modes of transportation using an approach based on both cellular automata and agent systems models to simulate what happens when pedestrians and other entities, such as cars, share spaces on the road. From this, the effects of shared spaces on traffic flow can be observed and analysed, allowing an evaluation of slower designs and more accident-prone areas.
\end{abstract}